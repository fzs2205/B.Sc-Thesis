\section{Conclusion}
Earlier, in the field of automated diagnosis, ASD identification was questionnaire-based,
parents interview-based, ADOS score or video gesture-based. Biswal et al. in \cite{biswal2012resting} discovered
that various brain regions still actively interact with each other while a subject was at rest (not
in any cognitive task) in 1995. Then onwards, rs-fMRI has evolved as a noteworthy tool to
explore brain networks by investigating local and global connectivity patterns and recent
research has generated many invaluable insights regarding factors and brain regions that
underlie brain. The primary goal of psychiatric neuroimaging research is to identify objective
biomarkers that may inform the diagnosis and treatment of brain-based disorders. However,
research works conducted using resting-state features for autism detection failed to reach
biomarker standard till now.\\

In this paper, a deep learning approach using multisite resting-state fMRI was intro-duced to predict ASD. ASD detection is a challenging task since no standard modeling choice has yet been recognized, and the current practice is very much diverse. In this pa-per, preprocessed fMRI data were obtained from the CPAC pipeline. To extract mean BOLD signals from preprocessed data, brain atlases were used. A single brain atlas that can serve as a biomarker for the detection of ASD has not yet been discovered. Thus, four different standard and predefined atlases were used to extract ROIs. Connectivity matrices were prepared using tangent embedding and flattened to form a feature vector removing redundant information. This feature vector was provided as input to our proposed model. Hidden layer configuration of the model was also varied, and its impact on detection observed. After performing a wide array of experiments, it has been confirmed that the BASC atlas using 122 ROIs yields higher predictive power than AAL, CC200, or Power atlases and can be considered to be more reliable in ASD diagnosis. It achieved 88\% accuracy, 90\% sensitivity, 87\% F1-score, and 96\% area under receiver operating characeristic curve. This result transcends most of the performances of existing works indicating that it is a promising method for ASD diagnosis. 

\section{Future Work}
The present study marked a significant performance improvement compared to existing studies. Despite that, some limitations need to be addressed. Only functional MRI data were utilized here for classification, whereas a combination of functional and structural MRI data has proven to achieve high prediction accuracy in \cite{rakic2020improving, mellema2019multiple}. Therefore, in future studies, other imaging modalities, such as structural MRI along with functional MRI data, may contain complementary information regarding ASD. However, implementation of the domain adaption technique \cite{li_gu_dvornek_staib_ventola_duncan_2020} and encoding decoding technique \cite{du2019brain} would also aid in prediction with more reliability and generalize well on unseen data obtained from different screening sites following different acquisition protocol. Other advanced neural network architectures, such as CNN, 3D based CNN model, etc., can also be utilized for prediction purposes and might prove to be fruitful. Furthermore, other options for implementing pipeline steps, such as usage of other available atlases, such as CC400, HO, Dosenbach, MSDL, etc., usage of first principal component-based time series extraction as in \cite{yassin2020machine}, non-correlation based functional connectivity matrix parametrization as in \cite{venkatesh2020comparing} and graph-based spectral method of vectorization as in \cite{farahani2019application}, are aimed to be implemented in future studies.